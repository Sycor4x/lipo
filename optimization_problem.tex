%tex

\documentclass[11pt]{article}
\usepackage[left=0.5in,top=0.5in,right=0.5in,foot=.5in,nohead]{geometry} 
\usepackage{amssymb, amsmath, amsthm, graphicx}

\usepackage{listings}
\usepackage{color}

\definecolor{dkgreen}{rgb}{0,0.6,0}
\definecolor{gray}{rgb}{0.5,0.5,0.5}
\definecolor{mauve}{rgb}{0.58,0,0.82}

\lstset{
  frame=tb,
  language=Python,
  aboveskip=3mm,
  belowskip=3mm,
  showstringspaces=false,
  columns=flexible,
  basicstyle={\small\ttfamily},
  numbers=none,
  numberstyle=\tiny\color{purple},
  keywordstyle=\color{blue},
  commentstyle=\color{gray},
  stringstyle=\color{gray},
  breaklines=true,
  breakatwhitespace=true,
  tabsize=3
}

\begin{document}

\begin{center}
  \textbf{David J. Elkind - LIPO} \\
\end{center}

I'm interested in estimating the Lipschitz constant $k$ for functions which approximate as Lipschitz. When the functions are noisy, they can't be Lipschitz, but adding a noise term $s$ can make the functions behave more nicely; including the large penalty $10^6$ for $s$ forces it to often be 0.

The function of interest is $f$ and it is only observed via some finite number of points $(x_i, f(x_i))$ for $i = 1,2,\dots t$.

Define
\begin{align}
U(y) &= \min_{i \in 1,2,\dots t} \left[ f(x_i) + \sqrt{s_i + (y - x_i)^\top K (y - x_i)} \right]
\end{align}

One way to estimate $k$ is to optimize the following program (using a different $k_i$ for each coordinate):
\begin{align}
  \min_{K, s}~~& ||K||_F^2 + 10^6 \sum_{i=1}^t s_i^2 \\
  \text{s.t.}~~& U(x_i) \ge f(x_i) i \in \{1,2, \dots t\} \\
  & s_i \ge 0 \forall i \in \{1,2, \dots t\} \\
  & K_{i,j} \ge 0 \forall i,j \in \{1, 2, \dots d\}^2 \\
  & \text{K is a diagonal matrix.} 
\end{align}

Expressing a signle constraint, we have 
\begin{align}
U(x_i) &\ge f(x_i) ~\forall i = 1, 2, \dots, t \\
\implies
f(x_j) + \sqrt{s_j + (x_i - x_j)^\top K (x_i - X_j)} &\ge f(x_i) \\
s_j + (x_i - x_j)^\top K (x_i - x_j) &\ge \left(f(x_i) - f(x_j)\right)^2\\
\end{align}
Which is linear in $s$ and $K$. Moreover, this is exactly the definition of Lipschitz continuity when $s_j =0 $ (hooray!).

This is a little bit of a bummer, though, since this isn't symmetric in $i,j$ due to the presence of the $s_j$. This implies that there are $t^2 > \binom{t}{2}$ constraints, which might be too many for large problems.

\end{document}