%tex

\documentclass[11pt]{article}
\usepackage[left=0.5in,top=0.5in,right=0.5in,foot=.5in,nohead]{geometry} 
\usepackage{amssymb, amsmath, amsthm, graphicx}

\usepackage{listings}
\usepackage{color}

\definecolor{dkgreen}{rgb}{0,0.6,0}
\definecolor{gray}{rgb}{0.5,0.5,0.5}
\definecolor{mauve}{rgb}{0.58,0,0.82}

\lstset{
  frame=tb,
  language=Python,
  aboveskip=3mm,
  belowskip=3mm,
  showstringspaces=false,
  columns=flexible,
  basicstyle={\small\ttfamily},
  numbers=none,
  numberstyle=\tiny\color{purple},
  keywordstyle=\color{blue},
  commentstyle=\color{gray},
  stringstyle=\color{gray},
  breaklines=true,
  breakatwhitespace=true,
  tabsize=3
}

\begin{document}

\begin{center}
  \textbf{David J. Elkind - LIPO} \\
\end{center}

I'm interested in estimating the Lipschitz constant $k$ for functions which approximate as Lipschitz. When the functions are noisy, they can't be Lipschitz, but adding a noise term $s$ can make the functions behave more nicely; including the large penalty $10^6$ for $s$ forces it to often be 0.

The function of interest is $f$ and it is only observed via some finite number of points $(x_i, f(x_i))$ for $i = 1,2,\dots t$.

Define
\begin{align}
U(y) &= \min_{i \in 1,2,\dots t} \left[ f(x_i) + \sqrt{s_i + (y - x_i)^\top K (y - x_i)} \right]
\end{align}

One way to estimate $k$ is to optimize the following program (using a different $k_i$ for each coordinate):
\begin{align}
  \min_{K, s}~~& ||K||_F^2 + 10^6 \sum_{i=1}^t s_i^2 \\
  \text{s.t.}~~& U(x_i) \ge f(x_i) i \in \{1,2, \dots t\} \\
  & s_i \ge 0 \forall i \in \{1,2, \dots t\} \\
  & K_{i,j} \ge 0 \forall i,j \in \{1, 2, \dots d\}^2 \\
  & \text{K is a diagonal matrix.} 
\end{align}

This clearly can be rewritten as 

\begin{align}
  \min_{k, s}~~& k^\top k + 10^6 \cdot s^\top s \\
  \text{s.t.}~~& U(x_i) \ge f(x_i) ~ i \in \{1,2, \dots t\} \\
  & s_i \ge 0 ~ \forall i \in \{1,2, \dots t\} \\
  & k_{i} \ge 0 ~ \forall i \in \{1, 2, \dots d\}^2 
\end{align}

But it's not at all clear how to me actually \textit{do} the optimization, since the  minimization objective is not obviously a QP. I guess one way to do it would be to roll the two summands together and write

\begin{align}
  \min_{k, s}~~& \begin{bmatrix}k \\ s \end{bmatrix}^\top A \begin{bmatrix}k \\ s \end{bmatrix} \\
  \text{s.t.}~~& U(x_i) \ge f(x_i) ~ i \in \{1,2, \dots t\} \\
  & s_i \ge 0 ~ \forall i \in \{1,2, \dots t\} \\
  & k_{i} \ge 0 ~ \forall i \in \{1, 2, \dots d\}^2 
\end{align}

Just for convenience, we can rewrite $U$ in terms of $k$:
\begin{align}
U(y) &= \min_{i \in 1,2,\dots t} \left[ f(x_i) + \sqrt{s_i + ||k^\top (y - x_i)||_2^2} \right]
\end{align}
and observe that all I've done is move stuff around so that $A$ is given by 
\begin{align}
  A_{ij} &=
  \begin{cases}
    1 &\forall i = j \land 1 \le i \le d \\
    10^6 &\forall i = j \land d < i \le t + d \\
    0 &\text{otherwise}
  \end{cases},
\end{align}
i.e. $A$ is a diagonal matrix with 1s corresponding to the elements of $k$ and $10^6$ corresponding to the elements of $s$.

This is ``nicer'' in a sense, since it's \textit{obviously} a QP. But what's less-nice is that $U$, and therefore its corresponding constraint, is not a linear function, owing to the fact that $s$ and $k$ both appear inside of the square root.

So given that this program is not a QP with linear constraints, what would the appropriate tool be to solve this problem? More specifically, what python library would you recommend using?
\end{document}